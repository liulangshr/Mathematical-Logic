%定义文档类型和基本的页面设置
\documentclass[12pt,a4paper]{article}

%加载要用到的宏包
\usepackage[utf8]{inputenc}
\usepackage{amsmath}
\usepackage{amsfonts}
\usepackage{amssymb}
\usepackage{amsthm}
\usepackage{bussproofs}
\usepackage{geometry}   %设置页边距的宏包

\geometry{left=1cm,right=1cm,top=2.5cm,bottom=2.5cm} 

%自定义环境
\theoremstyle{plain}
\newtheorem{exercise}{Exercise}

%文档的标题
\title{Homework 3}

%文档的作者
\author{\\
}

%文档的日期
\date{Deadline for Ex.\ 1 to Ex.\ 6: 23:59, October 25th, 2018
\\
Deadline for Ex.\ 7 to Ex.\ 9: 23:59, November 1st, 2018}

%文档内容开始
\begin{document}

%打出文档的标题、作者、日期等信息
\maketitle

Please do Exercises 2.1.3, 2.2.1, 2.3.2, 2.4.1 and 2.4.2 in the textbook and compare your solutions to those at the end of the book.

You do NOT need to hand in your solutions to the above exercises.

\ \\
\begin{exercise}
Please solve Exercise 2.4.3 in the textbook.
(You just need to write down the answers; there is no need to explain or to give reasons.)

Each of the following derivations proves a sequent. 
Write out the sequent that it proves.

\begin{itemize}

\item[(a)] \
%
\begin{prooftree}
\AxiomC{$[ \phi ]^1$}
\RightLabel{($\rightarrow$I)}
\UnaryInfC{$( \psi \rightarrow \phi )$}
\RightLabel{($\rightarrow$I),1}
\UnaryInfC{$( \phi \rightarrow ( \psi \rightarrow \phi ) )$}
\end{prooftree}
%
\item[(b)] \

\begin{prooftree}
\AxiomC{$\phi$}
\RightLabel{($\rightarrow$I)}
\UnaryInfC{$( \psi \rightarrow \phi )$}
\RightLabel{($\rightarrow$I)}
\UnaryInfC{$( \phi \rightarrow ( \psi \rightarrow \phi ) )$}
\end{prooftree}

\item[(c)] \

\begin{prooftree}
\AxiomC{$[ \psi ]^1$}
\AxiomC{$( \phi \wedge \psi )$}
\RightLabel{($\wedge$)E}
\UnaryInfC{$\phi$}
\RightLabel{($\wedge$I)}
\BinaryInfC{$(\psi \wedge \phi )$}
\RightLabel{($\rightarrow$I),1}
\UnaryInfC{$( \psi \rightarrow (\psi \wedge \phi ) )$}
\end{prooftree}

\item[(d)] \

\begin{prooftree}
\AxiomC{$[ \phi ]^1$}
\RightLabel{($\rightarrow$I),1}
\UnaryInfC{$( \phi \rightarrow \phi )$}
\end{prooftree}

\end{itemize}
\end{exercise}

\begin{proof}[Solution]\ 
    \begin{enumerate}
        \item $ \vdash ( \phi \rightarrow ( \psi \rightarrow \phi ) )$
        \item $\phi \vdash ( \phi \rightarrow ( \psi \rightarrow \phi ) )$
        \item $(\phi \wedge \phi) \vdash ( \psi \rightarrow (\psi \wedge \phi ) )$
        \item $\vdash ( \phi \rightarrow \phi )$
    \end{enumerate}
\end{proof}

\ \\
\begin{exercise}
Please solve (c) - (f) and (h) in Exercise 2.4.4 in the textbook:

Write out derivations to prove each of the following sequents.
%
\begin{itemize}

\item[(c)] $\vdash ( ( \phi \rightarrow ( \theta \rightarrow \psi ) ) \rightarrow ( \theta \rightarrow ( \phi \rightarrow \psi ) ) )$.

\item[(d)] $\{ ( \phi \rightarrow \psi ) , ( \phi \rightarrow \chi) \} \vdash ( \phi \rightarrow ( \psi \wedge \chi ))$.

\item[(e)] $\{ ( \phi \rightarrow \psi ) , ( ( \phi \wedge \psi ) \rightarrow \chi ) \} \vdash ( \phi \rightarrow \chi )$.

\item[(f)] $\{ ( \phi \rightarrow ( \psi \rightarrow \chi )) \} \vdash (( \phi \wedge \psi) \rightarrow \chi)$.

%\item[(g)] $\vdash (( \phi \rightarrow \psi ) \rightarrow (( \psi \rightarrow \theta ) \rightarrow ( \phi \rightarrow \theta )))$.

\item[(h)] $\vdash (( \phi \rightarrow ( \psi \wedge \theta )) \rightarrow (( \phi \rightarrow \theta ) \wedge ( \phi \rightarrow \psi )))$.

\end{itemize}
\end{exercise}

\begin{proof}\
    \begin{enumerate}
        \item[(c)]\ 
            \begin{prooftree}
                \AxiomC{$[\theta]^2$}
                \AxiomC{$[\phi]^1$}
                \AxiomC{$[(\phi \to (\theta \to \psi))]^3$}
                \RightLabel{$(\to E)$}
                \BinaryInfC{$(\theta \to \psi)$}
                \RightLabel{$(\to E)$}
                \BinaryInfC{$\psi$}
                \RightLabel{$(\to I)$,1}
                \UnaryInfC{$(\phi \to \psi)$}
                \RightLabel{$(\to I)$,2}
                \UnaryInfC{$(\theta \to (\phi \to \psi))$}
                \RightLabel{($\to I$),3}
                \UnaryInfC{$(((\phi \to (\theta \to \psi))\to (\theta \to (\phi \to \psi)))$}
            
                
             \end{prooftree}
        \item[(d)]\ 
            \begin{prooftree}
                \AxiomC{$[\phi]^1$}
                \AxiomC{$(\phi \to \psi)$}
                \RightLabel{($\to E$)}
                \BinaryInfC{$\psi$}
                \AxiomC{$[\phi]^1$}
                \AxiomC{$(\phi \to \chi)$}
                \RightLabel{($\to E$)}
                \BinaryInfC{$\chi$}
                \RightLabel{($\wedge I$)}
                \BinaryInfC{$\psi \wedge \chi$}
                \RightLabel{($\to I$),1}
                \UnaryInfC{$(\phi \to (\psi \wedge \chi))$}
            \end{prooftree}
        \item[(e)]\
            \begin{prooftree}
                \AxiomC{$[\phi]^1$}
                \AxiomC{$[\phi]^1$}
                \AxiomC{$(\phi \to \psi)$}
                \RightLabel{($\to E$)}
                \BinaryInfC{$\psi$}
                \RightLabel{($\wedge I$)}
                \BinaryInfC{$(\phi \wedge \psi)$}
                \AxiomC{$((\phi \wedge \psi ) \to \chi))$}
                \RightLabel{($\to E$)}
                \BinaryInfC{$\chi$}
                \RightLabel{($\to I$),1}
                \UnaryInfC{$(\phi \to \chi)$}
            \end{prooftree}
        \item[(f)]\
            \begin{prooftree}
                \AxiomC{$[\phi \wedge \psi]^1$}
                \RightLabel{($\wedge E$)}
                \UnaryInfC{$\psi$}
                \AxiomC{$[\phi \wedge \psi]^1$}
                \RightLabel{($\wedge E$)}
                \UnaryInfC{$\phi$}
                \AxiomC{$(\phi \to (\psi \to \chi))$}
                \RightLabel{($\to E$)}
                \BinaryInfC{$(\psi \to \chi)$}
                \RightLabel{($\to E$)}
                \BinaryInfC{$\chi$}
                \RightLabel{$(\to I)$,1}
                \UnaryInfC{$((\phi \wedge \psi) \to \chi)$}
            \end{prooftree}
        \item[(h)]\
            \begin{prooftree}
                \AxiomC{$[\phi]^1$}
                \AxiomC{$[(\phi \to (\psi \wedge \theta))]^2$}
                \RightLabel{($\to E$)}
                \BinaryInfC{$\psi \wedge \theta$}
                \RightLabel{$(\wedge E)$}
                \UnaryInfC{$\theta$}
                \RightLabel{($\to I$)}
                \UnaryInfC{$(\phi \to \theta)$}
                \AxiomC{$[\phi]^1$}
                \AxiomC{$[(\phi \to (\psi \wedge \theta))]^2$}
                \RightLabel{($\to E$)}
                \BinaryInfC{$\psi \wedge \theta$}
                \RightLabel{$(\wedge E)$}
                \UnaryInfC{$\psi$}
                \RightLabel{($\to I$),1}
                \UnaryInfC{$(\phi \to \psi)$}
                \RightLabel{($\wedge I$)}
                \BinaryInfC{$(( \phi \rightarrow \theta ) \wedge ( \phi \rightarrow \psi )))$}
                \RightLabel{($\wedge I$),2}
                \UnaryInfC{$ (( \phi \rightarrow ( \psi \wedge \theta )) \rightarrow (( \phi \rightarrow \theta ) \wedge ( \phi \rightarrow \psi )))$}
            \end{prooftree}
        
    \end{enumerate}

\end{proof}

\ \\
\begin{exercise}
Please solve (a) - (c) in Exercise 2.5.1 in the textbook:

Give derivations to prove the following sequents:
%
\begin{itemize}

\item[(a)] $\{ \phi , ( \phi \leftrightarrow \psi ) \} \vdash \psi$.

\item[(b)] $\vdash ( \phi \leftrightarrow \phi )$.

\item[(c)] $\{ ( \phi \leftrightarrow \psi ) , ( \psi \leftrightarrow \chi ) \} \vdash ( \phi \leftrightarrow \chi )$.

\end{itemize}
\end{exercise}

\begin{proof}\
    \begin{enumerate}
        \item[(a)]\
            \begin{prooftree}
                \AxiomC{$\phi$}
                \AxiomC{$(\phi \leftrightarrow \psi)$}
                \RightLabel{($\leftrightarrow E$)}
                \UnaryInfC{$(\phi \to \psi)$}
                \RightLabel{($\to E$)}
                \BinaryInfC{$\psi$}
            \end{prooftree}   
        \item[(b)]\ 
            \begin{prooftree}
                \AxiomC{$[\phi]^1$}
                \RightLabel{$(\to I)$}
                \UnaryInfC{$\phi \to \phi$}
                \AxiomC{$[\phi]^1$}
                \RightLabel{$(\to I)$,1}
                \UnaryInfC{$\phi \to \phi$}
                \RightLabel{($\leftrightarrow I$)}
                \BinaryInfC{$\phi \leftrightarrow \phi$}
            \end{prooftree}
        \item[(c)]\
            \begin{prooftree}
                \AxiomC{$[\phi]^1$}
                \AxiomC{$(\phi \leftrightarrow \psi)$}
                \RightLabel{$(\leftrightarrow E)$}
                \UnaryInfC{$(\phi \to \psi)$}
                \RightLabel{$(\to E)$}
                \BinaryInfC{$\psi$}
                \AxiomC{$\psi \leftrightarrow \chi$}
                \RightLabel{$(\leftrightarrow E)$}
                \UnaryInfC{$(\psi \to \chi)$}
                \RightLabel{$(\to E)$}
                \BinaryInfC{$\chi$}
                \RightLabel{$(\to I)$,1}
                \UnaryInfC{$\phi \to \chi$}
                \AxiomC{$[\chi]^2$}
                \AxiomC{$(\psi \leftrightarrow \chi)$}
                \RightLabel{$(\leftrightarrow E)$}
                \UnaryInfC{$(\chi \to \psi)$}
                \RightLabel{$(\to E)$}
                \BinaryInfC{$\psi$}
                \AxiomC{$\psi \leftrightarrow \chi$}
                \RightLabel{$(\leftrightarrow E)$}
                \UnaryInfC{$(\psi \to \phi)$}
                \RightLabel{$(\to E)$}
                \BinaryInfC{$\phi$}
                \RightLabel{$(\to I)$,2}
                \UnaryInfC{$\chi \to \phi$}
                \RightLabel{$(\leftrightarrow I)$}
                \BinaryInfC{$\phi \leftrightarrow \chi$}
            \end{prooftree}
            
    \end{enumerate}
\end{proof}

\ \\
\begin{exercise}
Please solve Exercise 2.5.2 in the textbook:

Let $S$ be any set of statements, and let ${\sim}$ be the relation on $S$ defined by: for all $\phi , \psi \in S$,
%
\[
\phi \sim \psi \mbox{ if and only if } \vdash ( \phi \leftrightarrow \psi ).
\]
Show that $\sim$ is an equivalence relation on $S$.
\end{exercise}

\begin{proof}\ 
    \begin{enumerate}
        \item[a.] $\sim$ is reflexive:\\
        From exercise 3b above, we have $\vdash (\phi \to \phi)$.
        Hence, for every $\phi \in S$, $\phi \sim \phi$.
        
        \item[b.] $\sim$ is transitive:\\
        For all $\phi$,$\psi$ and $\chi$ in $S$, if $\phi \sim \psi$ and $\psi \sim \chi$, then $\vdash ( \phi \leftrightarrow \psi )$ and $\vdash ( \psi \leftrightarrow \chi )$. From exercise 3c, we have $\vdash ( \phi \leftrightarrow \chi )$. Hence, $\phi \sim \chi$.
        
        \item[c.] $\sim$ is symmetric:\\
        For all $\phi$ and $\psi$ in $S$, if $\phi \sim \psi$, then $\vdash (\phi \leftrightarrow \psi) $. We prove $\psi \leftrightarrow \phi$ by this tree:
            \begin{prooftree}
                \AxiomC{$(\phi \leftrightarrow \psi)$}
                \RightLabel{$(\leftrightarrow E)$}
                \UnaryInfC{$(\psi \to \phi)$}
                \AxiomC{$(\phi \leftrightarrow \psi)$}
                \RightLabel{$(\leftrightarrow E)$}
                \UnaryInfC{$(\phi \to \psi)$}
                \RightLabel{$(\leftrightarrow I)$}
                \BinaryInfC{$\psi \leftrightarrow \phi$}
            \end{prooftree}
        Thus, $\psi \sim \phi$.
        
        \item[d.] From $a$, $b$ and $c$, $\sim$ is an equivalence.
    \end{enumerate}
\end{proof}

\ \\
\begin{exercise}
Please solve Exercise 2.5.4 in the textbook:

Devise suitable sequent rules for $\leftrightarrow$.
\end{exercise}

\begin{proof}[Solution]\
    \begin{enumerate}
        \item[a.] Sequent Rule $(\leftrightarrow E)$:\\
        If $(\Gamma \vdash (\phi \leftrightarrow \psi))$ is a correct sequent, then the sequent $(\Gamma \vdash (\phi \to \psi))$ and $(\Gamma \vdash (\psi \to \phi))$ are both correct.
        
        \item[b.] Sequent Rule $(\leftrightarrow I)$:\\
        If the sequent $(\Gamma \vdash (\phi \to \psi))$ and $(\Gamma \vdash (\psi \to \phi))$ are both correct, then $(\Gamma \vdash (\phi \leftrightarrow \psi))$ is a correct sequent.
    \end{enumerate}

\end{proof}

\ \\
\begin{exercise}
Please solve (a), (e) and (f) in Exercise 2.6.1 in the textbook:

Find natural deduction proofs for the following sequents (none of which need (RAA)):
%
\begin{enumerate}

\item[(a)] $\vdash ( \neg ( \phi \wedge ( \neg \phi ) ) )$.

\item[(e)] $\{ ( \phi \rightarrow \psi ) \} \vdash ( ( \neg \psi ) \rightarrow ( \neg \phi ) )$.

\item[(f)] $\{ ( \phi \rightarrow \psi ) \} \vdash ( \neg ( \phi \wedge ( \neg \psi ) ) )$.

\end{enumerate}
\end{exercise}

\begin{proof}[Solution]\ 
    \begin{enumerate}
        \item[(a)]\ 
            \begin{prooftree}
                \AxiomC{$[(\phi \wedge (\neg \phi ))]^1$}
                \RightLabel{$(\wedge E)$}
                \UnaryInfC{$\phi$}
                \AxiomC{$[(\phi \wedge (\neg \phi ))]^1$}
                \RightLabel{$(\wedge E)$}
                \UnaryInfC{$\neg \phi$}
                \RightLabel{$(\neg E)$}
                \BinaryInfC{$\bot$}
                \RightLabel{$(\neg I)$,1}
                \UnaryInfC{$(\neg(\phi \wedge (\neg \phi )))$}
            \end{prooftree}
        \item[(e)]\
            \begin{prooftree}
                \AxiomC{$[\neg \psi]^2$}
                \AxiomC{$[\phi]^1$}
                \AxiomC{$(\phi \to \psi)$}
                \RightLabel{($\to I$)}
                \BinaryInfC{$\psi$}
                \RightLabel{$(\neg E)$}
                \BinaryInfC{$\bot$}
                \RightLabel{$(\neg I)$,1}
                \UnaryInfC{$\neg (\phi)$}
                \RightLabel{$(\to I)$,2}
                \UnaryInfC{$( ( \neg \psi ) \rightarrow ( \neg \phi ) )$}
            \end{prooftree}
        \item[(f)]\ 
            \begin{prooftree}
                \AxiomC{$[\phi \wedge \neg \psi]^1$}
                \RightLabel{$(\wedge E)$}
                \UnaryInfC{$\phi$}
                \AxiomC{$(\phi \to \psi)$}
                \RightLabel{$(\to E)$}
                \BinaryInfC{$\psi$}
                \AxiomC{$[\phi \wedge \neg \psi]^1$}
                \RightLabel{$(\wedge E)$}
                \UnaryInfC{$\neg \psi$}
                \RightLabel{$(\neg E)$}
                \BinaryInfC{$\bot$}
                \RightLabel{$(\neg I)$,1}
                \UnaryInfC{$( \neg ( \phi \wedge ( \neg \psi ) ) )$}
                
            \end{prooftree}
    \end{enumerate}

\end{proof}
%文档内容结束
\end{document}
