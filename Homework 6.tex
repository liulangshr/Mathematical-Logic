%定义文档类型和基本的页面设置
\documentclass[12pt,a4paper]{article}

%加载要用到的宏包
\usepackage[utf8]{inputenc}
\usepackage{amsmath}
\usepackage{amsfonts}
\usepackage{amssymb}
\usepackage{amsthm}
\usepackage{bussproofs}
\usepackage{tikz}
\usepackage[left=1.5cm,right=1.5cm]{geometry}

%自定义环境
\theoremstyle{plain}
\newtheorem{exercise}{Exercise}

%文档的标题
\title{Homework 6}

%文档的作者
\author{(Please write your name here)\\
(Please write your student number here)}

%文档的日期
\date{Deadline: 23:59, November 15th, 2018}

%文档内容开始
\begin{document}

%打出文档的标题、作者、日期等信息
\maketitle

%Please do Exercises 3.3.10 (b), 3.4.6(b) and 3.4.6 (d) in the textbook and compare your solutions to those at the end of the book.

%You do NOT need to hand in your solutions to the above exercises.

\begin{exercise}
Please write out truth tables for the formulas in the following list.
Then, for each of these formulas, please say whether it is (a) a tautology, (b) a contradiction, (c) satisfiable. 
(It can be more than one of these.)
%
\begin{itemize}

\item[(a)] $( ( ( \neg p ) \rightarrow q ) \rightarrow ( p \rightarrow ( \neg q ) ) )$.

\item[(b)] $( p \rightarrow ( ( \neg p ) \rightarrow q ) )$.

\item[(c)] $( ( \neg p ) \leftrightarrow p )$.

\end{itemize}
\end{exercise}

\begin{proof}
For (a):
\begin{center}
\begin{tabular}{cc||rlclcrcrl}
$p$ & $q$ & $( ( ( \neg$ & $p )$ & $\rightarrow$ & $q )$ & $\rightarrow$ & $( p$ & $\rightarrow$ & $( \neg$ &  $q ) ) )$ \\
\hline
T & T & F & T & T & T & F & T & F & F & T \\
T & F & F & T & T & F & T & T & T & T & F \\
F & T & T & F & T & T & T & F & T & F & T \\
F & F & T & F & F & F & T & F & T & T & F \\
  &   &  &   &  &   & $\Uparrow$ & &  & & 
\end{tabular}
\end{center}
%
Therefore, $( ( ( \neg p ) \rightarrow q ) \rightarrow ( p \rightarrow ( \neg q ) ) )$ is satisfiable.

\ \\
For (b):
\begin{center}
\begin{tabular}{cc||rcrlcl}
$p$ & $q$ & $( p$ & $\rightarrow$ & $( ( \neg$ & $p )$ & $\rightarrow$ &  $q ) )$ \\
\hline
T & T & T & T & F & T & T & T \\
T & F & T & T & F & T & T & F \\
F & T & F & T & T & F & T & T \\
F & F & F & T & T & F & F & F \\
  &   &  & $\Uparrow$  &  &   &  & 
\end{tabular}
\end{center}
%
Therefore, $( p \rightarrow ( ( \neg p ) \rightarrow q ) )$ is a tautology.

\ \\
For (c):
\begin{center}
\begin{tabular}{c||rlcl}
$p$ & $( ( \neg$ & $p )$ & $\leftrightarrow$ & $p )$ \\
\hline
T & F & T & F & T \\
F & T & F & F & F \\
& & & $\Uparrow$ &
\end{tabular}
\end{center} 
%
Therefore, $( ( \neg p ) \rightarrow p )$ is a contradiction.
\end{proof}

\ \\
\begin{exercise}
Please prove the following by truth tables:
%
\begin{itemize}

\item[(a)] $( p_1 \vee ( p_2 \wedge p_3 ) )$ eq $( ( p_1 \vee p_2 ) \wedge ( p_1 \vee p_3 ) )$

\item[(b)] $( \neg ( p_1 \vee p_2 ) )$ eq $( ( \neg p_1 ) \wedge ( \neg p_2 ) )$

\item[(c)] $( p \rightarrow q )$ eq $( ( \neg p ) \vee q )$

\item[(d)] $( p \rightarrow q )$ eq $( \neg ( p \wedge ( \neg q ) ) )$

\item[(e)] $( \neg ( p_1 \leftrightarrow p_2 ) )$ eq $( ( \neg p_1 ) \leftrightarrow p_2 )$

\end{itemize} 
\end{exercise}

\begin{proof}
For (a):
%
\begin{center}
\begin{tabular}{ccc||rcrcl|rclcrcl}
$p_1$ & $p_2$ & $p_3$ & $( p_1$ & $\vee$ & $( p_2$ & $\wedge$ & $p_3 ) )$ & $( ( p_1$ & $\vee$ & $p_2 )$ & $\wedge$ & $( p_1$ & $\vee$ & $p_3 ) )$ \\
\hline
T & T & T & T & T & T & T & T & T & T & T & T & T & T & T \\
T & T & F & T & T & T & F & F & T & T & T & T & T & T & F \\
T & F & T & T & T & F & F & T & T & T & F & T & T & T & T \\
T & F & F & T & T & F & F & F & T & T & F & T & T & T & F \\
F & T & T & F & T & T & T & T & F & T & T & T & F & T & T \\
F & T & F & F & F & T & F & F & F & T & T & F & F & F & F \\
F & F & T & F & F & F & F & T & F & F & F & F & F & F & T \\
F & F & F & F & F & F & F & F & F & F & F & F & F & F & F \\
  &   &   &  & $\Uparrow$ & & & &  &  &  & $\Uparrow$ & & &
\end{tabular}
\end{center}

For (b):
%
\begin{center}
\begin{tabular}{cc||rrcl|rlcrl}
$p_1$ & $p_2$ & $( \neg$ & $( p_1$ & $\vee$ & $p_2 ) )$ & $(( \neg$ & $p_1 )$ & $\wedge$ & $( \neg$ & $p_2 ) )$ \\
\hline
T & T & F & T & T & T & F& T & F & F & T \\
T & F & F & T & T & F & F& T & F & T & F \\
F & T & F & F & T & T & T& F & F & F & T \\
F & F & T & F & F & F & T& F & T & T & F \\
  &   & $\Uparrow$ &  &  &  &  &  & $\Uparrow$ &  &
\end{tabular}
\end{center}

For (c):
%
\begin{center}
\begin{tabular}{cc||rcl|rlcl}
$p$ & $q$ & $( p$ & $\rightarrow$ & $q )$ & $(( \neg$ & $p )$ & $\vee$ & $q )$ \\
\hline
T & T & T & T & T & F & T & T & T \\
T & F & T & F & F & F & T & F & F \\
F & T & F & T & T & T & F & T & T \\
F & F & F & T & F & T & F & T & F \\
  &   &  & $\Uparrow$ & &  &  & $\Uparrow$ & 
\end{tabular}
\end{center}

For (d):
%
\begin{center}
\begin{tabular}{cc||rcl|rrcrl}
$p$ & $q$ & $( p$ & $\rightarrow$ & $q )$ & $( \neg$ & $( p$ & $\wedge$ & $( \neg$ & $q )))$ \\
\hline
T & T & T & T & T & T & T & F & F & T \\
T & F & T & F & F & F & T & T & T & F \\
F & T & F & T & T & T & F & F & F & T \\
F & F & F & T & F & T & F & F & T & F \\
  &   &  & $\Uparrow$ & & $\Uparrow$ &  &  &  &
\end{tabular}
\end{center}

For (e):
%
\begin{center}
\begin{tabular}{cc||rrcl|rlcl}
$p_1$ & $p_2$ & $( \neg$ & $( p_1$ & $\leftrightarrow$ & $p_2 ))$ & $(( \neg$ & $p_1)$ & $\leftrightarrow$ & $p_2 )$ \\
\hline
T & T & F & T & T & T & F & T & F & T \\
T & F & T & T & F & F & F & T & T & F \\
F & T & T & F & F & T & T & F & T & T \\
F & F & F & F & T & F & T & F & F & F \\
  &   & $\Uparrow$ &  &  & & & & $\Uparrow$ &
\end{tabular}
\end{center}
\end{proof}

\begin{exercise}
Please prove the following version of Lemma 3.5.10:

\ \\
Let $\sigma$ and $\tau$ be signatures, $A$ and $B$ a $\sigma$-structure and a $\tau$-structure respectively. 
The following are equivalent:
%
\begin{itemize}

\item[(i)] $A(p) = B(p)$ holds for every propositional symbol $p \in \sigma \cap \tau$;

\item[(ii)] for each formula $\phi$ of LP($\sigma \cap \tau$), $A^* (\phi) = B^* (\phi)$. 

\end{itemize}
\end{exercise}

\begin{proof}\
    \begin{enumerate}
        \item\ 
            $(i) \Rightarrow (ii)$\\
            Let $\phi$ be an arbitrary formula of $LP(\sigma \cap \tau)$, and $\{p_1,p_2,..p_n\}$ be the set of all the propositional symbols in $\phi$.\\
            Obviously, for each $p$ s.t. $p \in P$, $p \in \sigma \cap \tau holds$.\\
            Then, for for each $p$ s.t. $p \in P$, $A(p) = B(p)$ holds.\\
            
    \end{enumerate}
\end{proof}

\ \\
\begin{exercise}
Please solve Exercise 3.7.1 in the textbook:

\ \\
Carry out the following substitutions.
%
\begin{itemize}

\item[(a)] $( p \rightarrow q)[p/q]$.

\item[(b)] $( p \rightarrow q)[p/q][q/p]$.

\item[(c)] $(p \rightarrow q)[p/q, q/p]$.

\item[(d)] $( r \wedge (p \vee q))[((t \rightarrow ( \neg p)) \vee q)/p, ( \neg ( \neg q))/q, (q \leftrightarrow p)/s]$.

\end{itemize}
\end{exercise}

\begin{proof}[Solution]
(Please write your solution here.)
\end{proof}

\ \\
\begin{exercise}
Please solve Exercise 3.7.2 (a) and (b) in the textbook:

\begin{itemize}

\item[(a)] Using one of the De Morgan Laws (Example 3.6.5), show how the following equivalences follow from the Replacement and Substitution Theorems:
%
\begin{align*}
( \neg ( ( p_1 \wedge p_2 ) \wedge p_3 ) ) \ &\text{eq} \ (( \neg ( p_1 \wedge p_2)) \vee (\neg p_3)) \\
&\text{eq}\ ((( \neg p_1) \vee ( \neg p_2)) \vee ( \neg p_3))
\end{align*}

\item[(b)] Show the following generalised De Morgan Law, by induction on $n$:
If $\phi_1 , \dots , \phi_n$ are any formulas then
%
\[
( \neg ( \dots ( \phi_1 \wedge \phi_2 ) \wedge \dots ) \wedge \phi_n )) \ \text{eq} \ ( \dots (( \neg \phi_1 ) \vee ( \neg \phi_2 ) ) \vee \dots ) \vee ( \neg \phi_n ))
\]
%
(And note that by the other De Morgan Law, the same goes with
$\wedge$ and $\vee$ the other way round.)

\end{itemize}
\end{exercise}

\begin{proof}
(Please write your solution here.)
\end{proof}

\ \\
\begin{exercise}
Please solve Exercise 3.7.6 in the textbook: (First, you may get some inspiration from (b) of Exercise 4 above.
Second, for simplicity, we may assume that we are working with a fixed signature $\sigma$ so that every formula that we discuss is a formula of LP($\sigma$), all of $S$, $T$ and the $U$ that you're going to define are \emph{total} functions on $\sigma$ and the formulas in the ranges of $S$, $T$ and $U$ are formulas of LP($\sigma$).
Please note that $S (p) = p$ means that we do not change $p$ when carrying out a substitution.)

\ \\
Suppose $S$ and $T$ are substitutions. 
Show that there is a substitution $U$ such that for every formula $\phi$, $\phi [U] = \phi [S] [T]$.
\end{exercise}

\begin{proof}
(Please write your proof here.)
\end{proof}

%文档内容结束
\end{document}