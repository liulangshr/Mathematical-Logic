%定义文档类型和基本的页面设置
\documentclass[12pt,a4paper]{article}

%加载要用到的宏包
\usepackage[utf8]{inputenc}
\usepackage{amsmath}
\usepackage{amsfonts}
\usepackage{amssymb}
\usepackage{amsthm}
\usepackage{bussproofs}
\usepackage{tikz}
%\usepackage[left=1.5cm,right=1.5cm]{geometry}

%自定义环境
\theoremstyle{plain}
\newtheorem{exercise}{Exercise}
\theoremstyle{definition}
\newtheorem{definition}{Definition}

%文档的标题
\title{Homework 8}

%文档的作者
\author{WEI Lai\\
1801211383}

%文档的日期
\date{Deadline: 23:59, November 30th, 2018}

%文档内容开始
\begin{document}

%打出文档的标题、作者、日期等信息
\maketitle

%Please do Exercises 3.3.10 (b), 3.4.6(b) and 3.4.6 (d) in the textbook and compare your solutions to those at the end of the book.

%You do NOT need to hand in your solutions to the above exercises.

In Exercises 1-3, we always use the signature $\sigma = \{ p_0 , p_1 , p_2 , \dots \}$, and thus we will omit it in the notations.
Moreover, we only consider formulas using at most the truth function symbols $\neg$, $\wedge$ and $\bot$.

In Exercises 1-3, we study the properties of special sets of formulas defined as follows:

\begin{definition}
A set $\Gamma$ of formula is a \emph{maximal consistent set}, if it satisfies both of the following conditions: 
%
\begin{enumerate}

\item $\Gamma$ is syntactically consistent, i.e.\ $\Gamma \not \vdash \bot$.

\item for each formula $\phi$, if $\phi \not \in \Gamma$, then $\Gamma \cup \{ \phi \}$ is \emph{not} syntactically consistent, i.e.\ $\Gamma \cup \{ \phi \} \vdash \bot$.
\ \\ 

\end{enumerate}
\end{definition}


\begin{exercise}
Let $\Gamma$ be a maximal consistent set of formulas, and $\phi$ and $\psi$ two formulas.
Please prove the following:
%
\begin{enumerate}

\item If $\Gamma \vdash \phi$, then $\phi \in \Gamma$.

\item $( \neg \phi ) \in \Gamma$, if and only if $\phi \not \in \Gamma$.

\item $( \phi \wedge \psi ) \in \Gamma$, if and only if $\phi \in \Gamma$ and $\psi \in \Gamma$.

\end{enumerate}  
%
(Hint: First, you do not need to use induction; second, you can use Exercise 3.4.6 without proof.)
\end{exercise}

\begin{proof}\
	\begin{enumerate}
    	\item
        Assume that $\Gamma \vdash \phi$ and \(\phi \notin \Gamma\). Since \(\phi \notin \Gamma\), by definition of maximal consistent,  $\Gamma \cup \{ \phi \} \vdash \bot$ holds. Since $\Gamma \vdash \phi$, by 3.4.6(d), \(\Gamma \vdash \bot \) holds. It contradicts with  $\Gamma \not \vdash \bot$.
    
        \item
        \begin{itemize}
            \item[$\Rightarrow$] 
            Assuem that $( \neg \phi ) \in \Gamma$ and $\phi \in \Gamma$. By 3.4.6(a), $\Gamma \vdash \neg \phi$ and $\Gamma \vdash \phi$ holds. Then $\Gamma \vdash \bot$ holds. It contradicts with  $\Gamma \not \vdash \bot$. 
            \item[$\Leftarrow$]
            Assume that $\phi \notin \Gamma$. By definition of maximal consistent,  $\Gamma \cup \{ \phi \} \vdash \bot$ holds. Then $\Gamma \cup \{ \phi \} \vdash \neg \phi$ holds. By 1, $\neg \phi \in \Gamma \cup \{ \phi \}$ holds. Then, $\neg \phi \in \Gamma $ holds.  
        \end{itemize}
        
        \item
        \begin{itemize}
            \item[$\Rightarrow$] 
            Assume that $(\phi \wedge \psi) \in \Gamma$. By 3.4.6(a), $\Gamma \vdash (\phi \wedge \psi)$. Thus, by $(\wedge I)$, $\Gamma \vdash \phi$ and $\Gamma \vdash \psi$ holds. By 1, $\phi \in \Gamma$ and $\psi \in \Gamma$ holds.
            \item[$\Leftarrow$]
            Assume that $\phi \in \Gamma$ and $\psi \in \Gamma$. By 3.4.6(a), $\Gamma \vdash \phi$ and $\Gamma \vdash \psi$ holds. Thus, by $(\wedge I)$, $\Gamma \vdash (\phi \wedge \psi)$ holds. By 1, $(\phi \wedge \psi) \in \Gamma$ holds.     
        \end{itemize}
	\end{enumerate}
	
\end{proof}

\ \\
\begin{exercise}
In this exercise, we are going to prove that every maximal consistent set has a model.

\ \\
Let $\Gamma$ be a maximal consistent set of formulas and $A : \sigma \rightarrow \{ \text{T} , \text{F} \}$ be a structure such that, for each $p \in \sigma$,
%
\[
A (p) = \left \{
\begin{array}{ll}
\text{T} , & \mbox{if } p \in \Gamma \\
\text{F} , & \mbox{if } p \not \in \Gamma
\end{array}
\right.
\]
%
Please prove that, for each formula $\phi$, $\phi \in \Gamma \Leftrightarrow A^* (\phi) = \text{T}$.
\end{exercise}

\begin{proof}\ \\
    Let $\phi$ be an arbitrary formula s.t. $\phi \in \Gamma$. It is easy to show that $\bot \notin \Gamma$. We use induction on $\phi$.
        \begin{itemize}
            \item[\textbf{BS:}]
                Assume that $\phi=p$.\\
                $\phi \in \Gamma$, iff $p \in \Gamma$. \hfill By $\phi = p$ and $\phi \in \Gamma$.\\
                \phantom{$\phi \in \Gamma$, }iff $A(p)=A^*(p)=A^*(\phi)=T$. \hfill By definition of $A(p)$.\\
            
            \item[\textbf{IH:}]
                For each formula $\psi$ and $\chi$ , $\psi \in \Gamma \Leftrightarrow A^* (\psi) = \text{T}$ and $\chi \in \Gamma \Leftrightarrow A^* (\chi) = \text{T}$.
            
            \item[\textbf{IS:}]
                \begin{itemize}
                    \item \textbf{Case 1:} 
                    Assume that $\phi= \neg \psi$.\\
                    $\phi \in \Gamma$, iff $\neg \psi \in \Gamma$. \hfill By $\phi= \neg \psi$ and $\phi \in \Gamma$. \\
                    \phantom{$\phi \in \Gamma$, }iff $\psi \notin \Gamma$. \hfill By Exercise 1.2.\\
                    \phantom{$\phi \in \Gamma$, }iff $A^*(\psi)=F$. \hfill By IH.\\
                    \phantom{$\phi \in \Gamma$, }iff $A^*(\neg \psi)=T$.\\
                    \phantom{$\phi \in \Gamma$, }iff $A^*(\phi)=T$.
                    
                    \item \textbf{Case 2:}
                    Assume that $\phi= \psi \wedge \chi$.\\
                    $\phi \in \Gamma$, iff $\psi \wedge \chi \in \Gamma$. \hfill By $\phi= \psi \wedge \chi$ and $\phi \in \Gamma$. \\  
                    \phantom{$\phi \in \Gamma$, }iff $\psi \in \Gamma$ and $\chi \in \Gamma$. \hfill By Exercise 1.3.\\ 
                    \phantom{$\phi \in \Gamma$, }iff $A^*(\psi)=T$ and $A^*(\chi)=T$. \hfill By IH.\\
                    \phantom{$\phi \in \Gamma$, }iff $A^*(\psi \wedge \chi)=T$.\\
                    \phantom{$\phi \in \Gamma$, }iff $A^*(\phi)=T$.
                \end{itemize}
        \end{itemize}
    This finishes the proof by induction.
\end{proof}

\ \\
\begin{exercise}
In this exercise, we are going to prove that every syntactically consistent set is a subset of some maximal consistent set.

This result is called \textbf{Lindenbaum's Lemma}.

\ \\ 
Let $\Gamma$ be a syntactically consistent set and $f : \mathbb{N} \rightarrow Form$ a bijection.
For convenience, we write $\phi_n$ for $f (n)$, for each $n \in \mathbb{N}$.
We define a sequence of formulas $\{ \Gamma_i \mid i \in \mathbb{N} \}$ as follows:
%
\begin{enumerate}

\item $\Gamma_0 = \Gamma$,

\item $
\Gamma_{i+1} = \left \{ 
\begin{array}{ll}
\Gamma_i \cup \{ \phi_i \} , & \mbox{if } \Gamma_i \cup \{ \phi_i \} \not \vdash \bot \\
\Gamma_i , & \mbox{if } \Gamma_i \cup \{ \phi_i \} \vdash \bot 
\end{array}
\right.
$
\end{enumerate}
%
Please prove the following:
%
\begin{enumerate}

\item For each $n \in \mathbb{N}$, $\Gamma_n$ is syntactically consistent.

\item $\Delta = \bigcup \{ \Gamma_i \mid i \in \mathbb{N} \}$ is a maximal consistent set.

\end{enumerate}
\end{exercise}

\begin{proof}\
\begin{enumerate}
    \item
    We use induction on $i$ s.t. $i \in \mathbb{N}$.
        \begin{itemize}
        
        \item[\textbf{BS:}]
            Suppose that $i=1$. Since $\Gamma_0 = \Gamma$ and $\Gamma$ is syntactically consistent, $\Gamma_0$ is syntactically consistent.
            
        \item[\textbf{IH:}]
            $\Gamma_i$ is syntactically consistent.
            
        \item[\textbf{IS:}]
                \begin{itemize}
                    \item \textbf{Case 1:} $\Gamma_i \cup \{\phi_i\} \vdash \bot $ holds.\\
                    Since $\Gamma_i \cup \{\phi_i\}\vdash \bot $, $\Gamma_{i+1}=\Gamma_i$ holds. By IH, $\Gamma_i$ is syntactically consistent. Then, $\Gamma_{i+1}$ is syntactically consistent.
                    
                    \item \textbf{Case 2:} $\Gamma_i \cup \{\phi_i\}  \not \vdash \bot $ holds.\\
                    Since $\Gamma_i \cup \{\phi_i\}  \not \vdash \bot $, $\Gamma_{i+1}=\Gamma_i \cup \{\phi_i\}$ and $\Gamma_i \cup \{\phi_i\}$ is syntactically consistent holds. Thus, $\Gamma_{i+1}$ is syntactically consistent.  
                \end{itemize}
        \end{itemize}   
    This finishes the proof by induction.
    
    \item
        \begin{enumerate}
            \item 
            We show that $\Delta$ is syntactically consistent.
            
            Suppose that $\Delta$ is not syntactically consistent, i.e. $\Delta \vdash \bot$.\\
            Since a derivation is finite, there must be a set $\Delta_0$ s.t. $\Delta_0 \subseteq \Delta$ and $\Delta_0 \vdash \bot$, i.e. $\Delta_0$ is not syntactically consistent.\\
            We choose $i$ s.t. $i=max\{j \vert \phi_j \in \Delta_0\}$.\\
            By definition of $\Gamma_n$, $\Delta_0 \subseteq \Gamma_i$ holds.\\
            Since $\Delta_0$ is not syntactically consistent, $\Gamma_i$ is not syntactically consistent. 
            By 1, $\Gamma_i$ is syntactically consistent. Contradiction!
            Thus, $\Delta$ is syntactically consistent.
            
            \item
            We show that for every forluma $\phi _i$, if $\phi_i \not \in \Delta$, then $\Delta \cup \{ \phi_i \}$ is not syntactically consistent, i.e.\ $\Delta \cup \{ \phi_i \} \vdash \bot$.
            
            Suppose that $\phi_i \not \in \Delta$ and $\Delta \cup \{ \phi_i \} \not \vdash \bot$.\\
            Since $\Delta \cup \{ \phi_i \} \not \vdash \bot$, $\Gamma_i \cup \{ \phi_i \} \not \vdash \bot$ holds. Because it is easy to show that if a set is syntactically consistent, then all of its subsets are syntactically consistent.\\
            By definition of $\Gamma_n$, $\Gamma_{i+1}=\Gamma_i \cup \{ \phi_i \}$ holds.\\
            Thus, $\phi_i \in \Gamma_{i+1}$ holds.\\
            Thus, $\phi_i \in \Delta$ holds. It contradicts with $\phi_i \not \in \Delta$.\\
            Therefore, if $\phi_i \not \in \Delta$, then $\Delta \cup \{ \phi_i \}$ is not syntactically consistent.
        \end{enumerate}
    Therefore, $\Delta$ is a maximal consistent set.
    
\end{enumerate}
\end{proof}

\ \\
\begin{exercise}
This exercise is about Hilbert-style calculus of propositional logic.
%
\begin{enumerate}

\item Please give a deduction which proves the sequent 
%
\[
\{ p , ( q \rightarrow ( p \rightarrow r )) \} \vdash^H ( q \rightarrow r )
\] 

\item Please prove the following sequent using the Deduction Theorem:
%
\[
\vdash^H ( ( \neg p ) \rightarrow ( p \rightarrow q ) )
\]

\end{enumerate}
\end{exercise}

\begin{proof}\
\begin{enumerate}
    \item 
        \begin{enumerate}
            \item $(p \to (q \to p))$ \hfill Instance of (A1)
            \item $p$ 
            \item $(q \to p)$ \hfill (a),(b) MP
            \item $((q \to (p \to r)) \to ((q \to p)(q \to r)))$ \hfill Instance of (A2)
            \item $(q \to (p \to r))$
            \item $((q \to p) \to (q \to r))$ \hfill (d),(e) MP
            \item $(q \to r)$ \hfill (f),(c) MP
        \end{enumerate}
        
    \item
        \begin{enumerate}
            \item $((\neg p) \to ((\neg q) \to (\neg p)))$ \hfill Instance of (A1)
            \item $(\neg p)$
            \item $((\neg q) \to (\neg p))$ \hfill (a),(b) MP
            \item $(((\neg q) \to (\neg p)) \to (p \to q))$ \hfill Instance of (A3)
            \item $(p \to q)$ \hfill (d),(c) MP
        \end{enumerate}
        Thus, $\{(\neg p)\} \vdash_H (p \to q)$ holds. By Deduction Theorem, \[ \vdash^H ( ( \neg p ) \rightarrow ( p \rightarrow q ) ) \] holds.
\end{enumerate}
\end{proof}

%文档内容结束
\end{document}